%$Id: quick_start.tex 19032 2012-06-21 08:30:14Z wdobler $
\documentclass[a4paper,12pt]{article}
\usepackage[utf8]{inputenc}
%\usepackage[spanish]{babel}
\usepackage{pslatex} %Para el pdf sea más mono.
\usepackage{eurosym}
\usepackage{amssymb}
\usepackage{latexsym}
\usepackage[dvips]{graphicx}
\usepackage{delarray}
\usepackage{amsmath}
%\usepackage{bbm}
%\usepackage{bbold}
%\usepackage{accents}
\usepackage{subfigure}
\usepackage{multirow}
\usepackage{fancyhdr}
%\usepackage{tocbibind} % Para que incluya la bibliografia en el indice
%\usepackage{bibtex}
\usepackage{wrapfig}
\usepackage{color}
\usepackage{hyperref}
%\usepackage{fmtcount}
\usepackage{parskip}
\frenchspacing

\graphicspath{{./fig/}{./png/}}

\setlength{\hoffset}{-1in}
\setlength{\textwidth}{7.5in}
\setlength{\voffset}{-1.2in}
\setlength{\textheight}{10.0in}

\title{Pencil-code: quick reference guide.}
%\dedicatory{pa mi}


\author{Illa R. Losada}
\date{}

\begin{document}
\maketitle

\tableofcontents

\newpage

\section{Download the Pencil Code}
The Pencil Code is an open source code. General information can be found at

\url{http://www.nordita.org/pencil-code/}.


You need \verb|svn| to download the latest version. 
%\begin{verbatim}
%svn checkout https://pencil-code.googlecode.com/svn/trunk/ pencil-code
%--username NAME
%\end{verbatim}


%where you replace NAME with your gmail name, and the password is 
%generated by pencil-code.googlecode.com, found under ($\rightarrow$profile
%$\rightarrow$settings).
 
The command 
 
\begin{verbatim}
svn checkout https://pencil-code.googlecode.com/svn/trunk/ pencil-code
\end{verbatim}

checks out the latest version of the code. More information can be found on
the website if you wish to make and commit changes to the code.

%where you replace NAME with your gmail name, and the password is 
%generated by pencil-code.googlecode.com, found under ($\rightarrow$profile
%$\rightarrow$settings). 
% changed so that instructions assume user has more general access and
% is not required to have committing rights - EC

An alternative older version of the pencil-code can be downloaded at
\url{
wget http://pencil-code.googlecode.com/files/pencil-code-r18525.tar.gz
}

The newly created \verb|pencil-code| directory contains several sub-directories
\begin{enumerate}
  \item \verb|samples|:  many sample problems
  \item \verb|doc|: a full-length manual \verb|manual.tex| and other documentation
  \item \verb|config|: all the configuration files.
%  \item \verb|doc| a very important directory containing the pencil-code
%    \verb|manual.tex|
  \item \verb|idl|: routines for visualization using IDL
  \item \verb|python|: routines for visualization using python
  \item \verb|src|: source code modules
\end{enumerate}

%added suggestions for lists of folders, removed double for 'doc' - EC

\section{Configure the shell}

%<<<<<<< .mine
%I the \verb|tcsh| shell, put in your \$HOME/.cshrc file the following lines:
%=======   Removed above, and rearranged order of instructions - EC
It is recommended to use cshell instead of bash.
Note: If you want to change your default \$SHELL to tcsh, type

\begin{verbatim}
  chsh
  /bin/tcsh
\end{verbatim}

You will be prompted for your normal unix password.

If using the tcsh shell, add to your \$HOME/.cshrc file the following lines:
%>>>>>>> .r19030
\begin{verbatim}

#
#  path for pencil code
#
if (! $?PENCIL_HOME) setenv PENCIL_HOME $HOME/pencil-code
if (-r $PENCIL_HOME/sourceme.csh) then
  set _sourceme_quiet; source $PENCIL_HOME/sourceme.csh; unset _sourceme_quiet
endif
\end{verbatim}


If using bash, add the following lines to your \$HOME/.bashrc file:
\begin{verbatim}
#
#  path for pencil code
#
if [ -z $PENCIL_HOME ]; then  export PENCIL_HOME=$HOME/pencil-code; fi
if [ -e $PENCIL_HOME/sourceme.sh ]; thenAlso add the following useful
  set _sourceme_quiet; source $PENCIL_HOME/sourceme.sh; unset _sourceme_quiet
fi
\end{verbatim}

You may find it useful to add the following alias:

With tcsh: \texttt{alias pc 'cd \$PENCIL\_HOME'} \\ %Added line break - EC
With bash: \texttt{alias pc='cd \$PENCIL\_HOME'}

so that you can directly move to the Pencil Code directory by typing 'pc'.

Source your updated \$HOME/.cshrc or \$HOME/.bashrc file by typing:

With tcsh: \texttt{source .cshrc} \\ %Added line break - EC
With bash: \texttt{source .bashrc}

\section{Configure makefile.}

In order to run the code the proper configuration file is needed. This file should contain all the information related to the compilers and their special options.

The configuration files are located in the \verb|config| directory and have names related to the host-id. The host-id is easily obtained by typing while in the pencil code directory the following:  
\begin{verbatim}
pc_build --debug
\end{verbatim}

Create a new config file in:
\texttt{config/hosts/user/HOST-ID.conf}
%changed the directory in hosts from Illa to user

The next step is customize this file, for example:
\begin{verbatim}
# Linux.conf
#
# Default settings for Linux systems
#
# $Id: quick_start.tex 19032 2012-06-21 08:30:14Z wdobler $

%section Makefile
# %include compilers/gfortran
 %include compilers/ifort
 %include compilers/mpif90
 %include compilers/gcc

FC=mpif90 
FFLAGS= -O3  
CC=mpicc 
CFLAGS=-O3 -DFUNDERSC=1 
LD_MPI= 
FFLAGS_DOUBLE=-r8 
%endsection Makefile

%section runtime
mpiexec=/usr/pkg/intel/2011.8.273/composer_xe_2011_sp1.8.273/mpirt/bin/intel64/m
pirun
%endsection runtime
# End of file
\end{verbatim}

You can use any file already in hosts as an example, preferably with a 
host-id similar to your own, or one recommended for your work.
%Might be simpler to suggest someone just use a configuration file already present
%or add a sample one to the directory for the pencil code itself - EC

\section{Useful commands.}
\begin{center}
\begin{tabular}{|l|l|}\hline
pc & move to the PC directory\\\hline
cd samples/conv-slab & move to the sample 'conv-slab' which is in a 'samples'
directory\\\hline
pc\_setupsrc & initialize the local 'src' directory, copy necessary files,
etc...\\\hline
make & 	compile the code\\\hline
mkdir data & create the data directory where all the outputs will be written
\\\hline
pc\_build & compiles the code according to the configuation file\\\hline
pc\_run & starts and runs the setup once\\\hline
start.csh & compute the initial setup at time t=0 \\\hline
run.csh & launch the main code that advances the equations in time\\\hline
\end{tabular}
\end{center}

\section{Configure the run}
%The structure of the Pencil code consists of different modules, containing
%different physics that you may or may not need for your problem. 
%Removed repeated sentence - EC

Either copy a directory from the sample setups or create a new one.

%Might be worth suggesting creating new directories to run from by copying from 
%samples, as opposed to running from the samples themselves. 

\subsection{Makefile}
In your new directory for your simulation, type
\begin{verbatim}
pc_setupsrc
\end{verbatim}
This creates the links to the root \verb|src| directory. Then type 
\begin{verbatim}
mkdir data
\end{verbatim}
to create a subdirectory \verb|data| that will contain the data produced by your simulation.

%changed from just pc_setupsrc. pc_setupsrc will not make the data directory - EC

Basic configuration files for the make:
\begin{verbatim}
src/Makefile.local
src/cparam.local
\end{verbatim}

The structure of the Pencil code consists of different modules, containing
different physics that you may or may not need for your problem. These are set
in \verb|Makefile.local|. 

Example of a file without mpi:
\begin{verbatim}
###                             -*-Makefile-*-
### Makefile for modular pencil code -- local part
### Included by `Makefile'
###

MPICOMM=nompicomm
#MPICOMM=mpicomm
GRAVITY=gravity_simple
EOS=eos_idealgas
FORCING=noforcing
ENTROPY=noentropy
MAGNETIC=magnetic
MAGNETIC_MEANFIELD=magnetic/meanfield
DENSITY=density
HYDRO=hydro
\end{verbatim}

This setup contains the physics for simple gravity, equations of state 
for ideal gas, no forcing, no entropy, magnetic fields, magnetic mean fields,
density, and hydrodynamics.

%added further clarification of sample so that it becomes obvious how physics
%modules are selected

The \texttt{src/cparam.local} file set the local settings concerning grid size
and number of CPUs. My file (no mpi):
\begin{verbatim}
!  -*-f90-*-  (for Emacs)    vim:set filetype=fortran:  (for vim)
!
!  cparam.local
!
!  Local settings concerning grid size and number of CPUs.
!  This file is included by cparam.f90
!
integer, parameter :: ncpus=1,nprocx=1,nprocy=1,nprocz=ncpus/(nprocx*nprocy)
integer, parameter :: nxgrid=128,nygrid=1,nzgrid=128
!
\end{verbatim}

Your CPUs will always be 1 unless using MPI, and if using MPI the processors should be selected according to the machine being used. Your grid size is the size of the experiment, in this case a 2-D field of 128 x 128.
%Probably write this better, but explain what's in the file so that the person
%can change it.

\subsection{Run configuration files}

There are three basic configuration files where all the physics and outputs are specified: print.in, run.in and start.in.
\begin{itemize}
 \item \verb|print.in|: list of desired output values and formatting
 \item \verb|run.in|: run time parameters values, number of timesteps, etc.
 \item \verb|start.in|: initialisation parameters
\end{itemize}

See Appendix J of the Pencil Code Manual for the complete list of possible parameters and values for each module. 
%Added reference to the manual -EC


%changed working in list -EC

\section{Running the code}

In order to use the code, you need to compile the code in each working directory. Basically there are two ways to compile and run the code. The only way of using the Makefile config file created before is using the new way:

%The standard way:    Redudant - EC
\begin{enumerate}
  \item \verb|pc_build| compile the code
  \item \verb|pc_run| run the code
    \begin{enumerate}
      \item \verb|pc_run start| can be run before \verb|pc_run|, to construct
        the initial condition.
    \end{enumerate}
\end{enumerate}

The old way of doing it was:
\begin{verbatim}
make
./start.csh
./run.csh
\end{verbatim}
But some options are usually needed to be specified by hand in the make command. If, for example, you wish to use mpi, you would add the following flag:
\begin{verbatim}
make FC=mpif90
\end{verbatim}

Once a simulation has been started, the run command will continue the run from the latest timestep and will not overwrite the data to start again from beginning. The only way to reset a run is to delete the \verb|data| directory, use \verb|pc_run start|, or \verb|./start.csh|. %Added the bit about running and rerunning - EC
    
\section{Setting up python.}
\subsection{Python modules requirements.}
The basic needed modules are: numpy and matplotlib.
\begin{itemize}
 \item numpy: all array definitions and operations.
  \item matplotlib: plotting.
\end{itemize}

Other really useful optional modules are: ipython and scipy.

\begin{itemize}
 \item ipython: enhanced python interpreter.
  \item scipy: science functions and utilities.
\end{itemize}


\subsection{Installation}
Untar the \texttt{tar.gz} file or go to the directory and simple type as root or sudoed:
\begin{verbatim}
python setup.py install
\end{verbatim}
For a user installation (no root permision):
\begin{verbatim}
python setup.py install --user
\end{verbatim}

\subsection{Using the module.}
Import the module:
\begin{verbatim}
import pencil as pc
\end{verbatim}
Some useful functions:
\begin{center}
\begin{tabular}{|l|l|}\hline
pc.read\_ts & Read ``time\_series.dat'' file. Parameters are added as members of the class. \\\hline
pc.read\_slices & read 2D slice binary files and return two arrays: one of (nslices,vsize,hsize) and other of time\\\hline
pc.animate\_interactive &  Assemble a 2D animation from a 3D array. \\\hline
%× & ×\\\hline
%× & ×\\\hline
%× & ×\\\hline
\end{tabular}
\end{center}

\section{Run a sample: an example}
% Searching in the pencil code
% 1D Jeans sample?
We will illustrate the workflow method 
Construct a folder in which you would like to run the sample
\begin{verbatim}
 > mkdir sample_test_jeans
\end{verbatim}

Now we copy the sample from the pencil code directory
\begin{verbatim}
 > pc_newrun ~/pencil-code/samples/1d-tests/jeans-x jeans-x
\end{verbatim}

The \verb|pc_setupsrc| command sets up the linking to the root \verb|src|
directory and makes a
\verb|data| directory that will contain the data produced by your simulation.
\begin{verbatim}
 /jeans-x> pc_setupsrc
\end{verbatim}

You can inspect the included modules in
\begin{verbatim}
 /jeans-x> vi src/Makefile.local 
\end{verbatim}
where I used the \verb|vi| text editor. The grid size can be inspected in 
\begin{verbatim}
 /jeans-x> vi src/cparam.local 
\end{verbatim}

The code is compiled by
\begin{verbatim}
 /jeans-x> pc_build
\end{verbatim}
and this may take some time.

Time to run the code. 
\begin{verbatim}
 /jeans-x> pc_run
\end{verbatim}

Visualizing the output can be either done with \verb|idl| or \verb|python|.

\subsection{IDL workflow}
Start \verb|idl|.
\begin{verbatim}
IDL>
\end{verbatim}
\end{document}
