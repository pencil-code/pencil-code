%% $Id$
%% This file was automatically generated by Pencil::DocExtractor,
%% so think twice before you modify it.
%%
%% Source files:
%%   hydro.f90
%%   chemistry.f90
%%   geometrical_types.f90
%%   gpu_astaroth.f90
%%   hydro_potential.f90
%%   noentropy.f90
%%   nogpu.f90
%%   nohydro.f90
%%   nopower_spectrum.f90
%%   noyinyang.f90
%%   noyinyang_mpi.f90
%%   particles_adsorbed.f90
%%   particles_chemistry.f90
%%   particles_surfspec.f90
%%   power_spectrum.f90
%%   test_chemistry.f90
%%   timestep.f90
%%   timestep_strang.f90
%%   timestep_subcycle.f90
%%   yinyang.f90
%%   yinyang_mpi.f90

% ---------------------------------------------------------------- %
\begin{longtable}{lp{0.9\textwidth}}
\toprule
  \multicolumn{1}{c}{\emph{Module}} & {\emph{Description}} \\
\midrule
  \var{hydro.f90} & This module takes care of most of the things related to velocity. \\
  \var{}          & Pressure, for example, is added in the energy (entropy) module. \\
\midrule
  \var{chemistry.f90} & This modules adds chemical species and reactions. \\
  \var{}          & The units used in the chem.in files are cm3,mole,sec,kcal and K \\
\midrule
  \var{geometrical_types.f90} & Collection of geometrical object types. \\
  \var{}          & (Presently only rectangular toroid) \\
\midrule
  \var{gpu_astaroth.f90} & This module contains GPU related types and functions to be used with the ASTAROTH nucleus. \\
\midrule
  \var{hydro_potential.f90} & This module takes care of most of the things related to velocity. \\
  \var{}          & Pressure, for example, is added in the energy (entropy) module. \\
\midrule
  \var{noentropy.f90} & Calculates pressure gradient term for \\
  \var{}          & polytropic equation of state $p=\text{const}\rho^{\Gamma}$. \\
\midrule
  \var{nogpu.f90} & This module contains GPU related dummy types and functions. \\
\midrule
  \var{nohydro.f90} & no variable $\uv$: useful for kinematic dynamo runs. \\
\midrule
  \var{nopower_spectrum.f90} & reads in full snapshot and calculates power spetrum of u \\
\midrule
  \var{noyinyang.f90} & This module contains Yin-Yang related dummy types and functions. \\
\midrule
  \var{noyinyang_mpi.f90} & This module contains Yin-Yang related dummy types and functions. \\
\midrule
  \var{particles_adsorbed.f90} & This module takes care of the evolution of adsorbed \\
  \var{}          & species on the particle surface for reactive particles \\
\midrule
  \var{particles_chemistry.f90} & This module implements reactive particles. \\
\midrule
  \var{particles_surfspec.f90} & immediate vicinity of reactive particles. \\
\midrule
  \var{power_spectrum.f90} & reads in full snapshot and calculates power spetrum of u \\
\midrule
  \var{test_chemistry.f90} & This modules adds chemical species and reactions. \\
  \var{}          & The units used in the chem.in files are cm3,mole,sec,kcal and K \\
\midrule
  \var{timestep.f90} & Runge-Kutta time advance, accurate to order itorder. \\
  \var{}          & At the moment, itorder can be 1, 2, or 3. \\
\midrule
  \var{timestep_strang.f90} & Runge-Kutta time advance, accurate to order itorder. \\
  \var{}          & At the moment, itorder can be 1, 2, or 3. \\
\midrule
  \var{timestep_subcycle.f90} & This is a highly specified timestep module currently only working \\
  \var{}          & together with the special module coronae.f90. \\
\midrule
  \var{yinyang.f90} & This module contains Yin-Yang related types and functions \\
  \var{}          & which are incompatible with FORTRAN 95. \\
\midrule
  \var{yinyang_mpi.f90} & This module contains Yin-Yang related types and functions \\
  \var{}          & which are incompatible with FORTRAN 95. \\
%
\bottomrule
\end{longtable}

