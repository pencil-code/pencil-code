%$Id: notes.tex,v 1.106 2024/01/17 02:11:12 brandenb Exp $
\documentclass{article}
%\documentclass[twocolumn]{article}
\setlength{\textwidth}{160mm}
\setlength{\oddsidemargin}{-0mm}
\setlength{\textheight}{220mm}
\setlength{\topmargin}{-8mm}
\usepackage{graphicx,natbib,bm,url,color}
\graphicspath{{./fig/}{./png/}}
\thispagestyle{empty}
\input macros
\def\red{\textcolor{red}}
\def\blue{\textcolor{blue}}
\title{}
\author{}
\date{\today,~ $ $Revision: 1.106 $ $}
\begin{document}
%\maketitle

%\section{Solving the constraint equation for $\EE$}
\section{Including the displacement current}

\subsection{Maxwell equations}

Amp\'ere's law with the displacement current $\partial\EE/\partial t$
included reads
\begin{equation}
\frac{\partial\EE}{\partial t}=\nab\times\BB-\JJ.
\label{dEdt}
\end{equation}
Here, we must obey the constraint equation for $\EE$,
\begin{equation}
\nab\cdot\EE=\rho_\mathrm{e}.
\label{constraint}
\end{equation}
We also have $\nab\cdot\BB=0$, which is readily obeyed by expressing
$\BB=\nab\times\AAA$ in terms of the magnetic vector potential $\AAA$
and solving for the uncurled induction equation
\begin{equation}
\frac{\partial\AAA}{\partial t}=-\EE-\nab\psi.
\label{dAdt}
\end{equation}

\subsection{Charge density and Ohm's law}

We compute $\rho_\mathrm{e}$ by solving the continuity equation for the
charge density.
It is obtained by taking the divergence of \Eq{dEdt}, i.e.,
\begin{equation}
\frac{\partial\rho_\mathrm{e}}{\partial t}=-\nab\cdot\JJ,
\label{drhoedt}
\end{equation}
which requires solving Ohm's law, i.e.,
\begin{equation}
\JJ=\sigma\left(\EE+\uu\times\BB\right),
\label{Ohm}
\end{equation}
so
\begin{equation}
\nab\cdot\JJ=\sigma\left[\nab\cdot\EE+\nab\cdot(\uu\times\BB)\right],
\label{divOhm}
\end{equation}
where the derivatives in
\begin{equation}
\nab\cdot(\uu\times\BB)=\epsilon_{ijk}\left(u_{j,i}B_k+u_j B_{k,i}\right)
\label{divuxB}
\end{equation}
can be expressed in terms of $u_{j,i}$ and $B_{k,i}$, which are readily
available.

\subsection{Coulomb gauge}

One way to satisfy \Eq{constraint} is to adopt the Coulomb
gauge, $\nab\cdot\AAA=0$.
By taking the divergence of \Eq{dAdt} and using the Coulomb gauge, we have
$\nabla^2\psi=-\nab\cdot\EE$, so we have to solve a Poisson equation
for the electrostatic (or scalar) potential $\psi$,
\begin{equation}
\nabla^2\psi=-\rho_\mathrm{e},
\end{equation}
which is analogous to the Poisson equation for the gravitational potential.

\subsection{Weyl gauge}

When using the Weyl gauge (also known as the temporal gauge),
we can control the longitudinal modes of $\EE$, by defining
\begin{equation}
\Gamma=\nab\cdot\AAA
\end{equation}
and replacing $\nab\times\BB$ in \Eq{dEdt} by $-\nabla^2\AAA+\nab\Gamma$,
so therefore \Eq{dEdt} becomes
\begin{equation}
\frac{\partial\EE}{\partial t}=-\nabla^2\AAA+\nab \Gamma-\JJ.
\label{dE2dt}
\end{equation}
The longitudinal modes of $\EE$ are constrained by
taking the divergence of \Eq{dAdt}, so we obtain
\begin{equation}
\frac{\partial\Gamma}{\partial t}=-\nab\cdot\EE.
\label{dGdt}
\end{equation}
Following standard procedures employed in numerical relativity,
and following a suggestion from Tanmay Vachaspati, we can
replace $\nab\cdot\EE$ by the algebraic mean of $\nab\cdot\EE$ and
$\rho_\mathrm{e}$, i.e., we solve
\begin{equation}
\frac{\partial\Gamma}{\partial t}=-(1-w)\nab\cdot\EE
-w\rho_\mathrm{e}.
\label{dG2dt}
\end{equation}

\section{Finite axion density}

When the axion density $\phi$ is finite, \Eq{dEdt} is replaced by
\begin{equation}
\frac{\partial\EE}{\partial t}=\nab\times\BB-\JJ
-\frac{\alpha}{f}\left(\dot{\phi}\BB+\nab\phi\times\EE\right).
\label{dEdt2}
\end{equation}
Now \Eq{dEdt2} becomes
\begin{equation}
\frac{\partial\EE}{\partial t}=-\nabla^2\AAA+\nab \Gamma-\JJ
-\frac{\alpha}{f}\left(\dot{\phi}\BB+\nab\phi\times\EE\right).
\label{dE2dt2}
\end{equation}
The longitudinal modes of $\EE$ are constrained by
\begin{equation}
\nab\cdot\EE=\rho_\mathrm{e}-\frac{\alpha}{f}\BB\cdot\nab\phi.
\end{equation}
Instead of \Eq{dGdt}, we now solve
\begin{equation}
\frac{\partial\Gamma}{\partial t}=-(1-w)\nab\cdot\EE
-w\rho_\mathrm{e}^\mathrm{tot},
\label{dG2dt2}
\end{equation}
where $\rho_\mathrm{e}^\mathrm{tot}\equiv\rho_\mathrm{e}-(\alpha/f)\,\BB\cdot\nab\phi$.
In the conducting case, we obtain $\rho_\mathrm{e}^\mathrm{tot}$ by solving the continuity
equation for the charge density.
It is obtained by taking the divergence of \Eq{dEdt2}, i.e.,
\begin{equation}
\frac{\partial\rho_\mathrm{e}^\mathrm{tot}}{\partial t}=-\nab\cdot\JJ^\mathrm{tot},
\label{drhoedt}
\end{equation}
where
\begin{equation}
\JJ^\mathrm{tot}=\JJ+\frac{\alpha}{f}\left(\dot{\phi}\BB+\nab\phi\times\EE\right).
\label{Jtot}
\end{equation}

In summary, when $\sigma=0$ and $\alpha\neq0$, we have to solve three dynamical equations:
\Eq{dAdt}, \eq{dE2dt2}, and \Eq{dG2dt2}.
When $\sigma=0$ and $\alpha=0$, the equations are linear and we just need
to solve the two dynamical equations \Eqs{dAdt}{dE2dt}, as was done in
BS21 and BHS21.
When $\sigma\neq0$, we also have to solve \Eq{drhoedt}.

\end{document}
